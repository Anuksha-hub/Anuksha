\documentclass[11pt]{article}
\usepackage{graphicx}
\usepackage{caption}
\begin{document}
\textbf{TOP 10 LARGEST STATES IN INDIA BY AREA }\\

\begin{tabular}{ c c c c }
 Rank & State          & Area     & Region\\
 \hline 
 1    & Rajasthan      & 342,239	 & Northern\\		
 2    & Madhya Pradesh & 308,245  & Central\\	
 3    & Maharashtra	   & 307,713  &	Western\\	
 4    & Uttar Pradesh  & 240,928   & Northern	\\	
 5    & Gujarat        & 196,024   &	Western\\	
 6    & Karnataka      & 191,791   & Southern\\
 7    & Andhra Pradesh & 162,975   & Southern	\\
 8    & Odisha	       & 155,707   & Eastern\\		
 9    & Chhattisgarh    & 135,191  & Central\\	
 10   & Tamil Nadu      & 130,058  & Southern \\
\end{tabular}\\

\begin{figure}[h]
\includegraphics[width=0.7\textwidth]{area.jpg} \caption{Largest states by area}
\label{fig:Map of India}
\end{figure} 
\pagebreak
\textbf{TOP 10 LARGEST STATES IN INDIA BY POPULATION}\\ 

\begin{tabular}{ c c c }
Rank & State & Population\\
\hline 
1    & Uttar Pradesh  & 19.98 Crore\\		
2    & Maharashtra & 11.24 Crore\\	
3    & Bihar	   & 10.41 Crore\\	
4    & West Bengal  & 9.13 Crore\\	
5    & Andhra Pradesh  & 8.46 Crore\\	
6    & Madhya Pradesh  & 7.26 Crore\\
7    & Tamil Nadu & 7.21 Crore\\
8    & Rajasthan  & 6.85 Crore\\		
9    & Karnataka  & 6.11 Crore\\	
10   & Gujarat    &   6.04 Crore\\
\end{tabular}\\

\end{document}
